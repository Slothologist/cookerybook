% Complete recipe example

\begin{cleanrecipe}
[
preparationtime = 20,
cookingtime = 180,
bakingtemperature = 80,
portion = 5,
author = Robert
]	
	{Zwiebeln schmoren für Dummies}
	
	\g{500}{Zwiebeln}
	
	\introduction{
		Stired onions are actually really easy to make, even though they are often not done properly and many people are having trouble getting them right and not burning them. In this recipe, it will be almost impossible to burn them, they will be delicious, but on the downside it will take a really long time to make them.
		}
		
	\preparation{
		Zwiebeln schälen und schneiden. Die Form ist nicht wirklich wichtig, auch wenn in den meisten Rezepten gewürfelte Zwiebeln zu bevorzugen sind. Die Dicke der Zwiebeln ändert nichts an der Kochzeit. %HINWEIS: Zur Übung sind Ringe zu bevorzugen
		
		Butter in einer großen Pfanne oder Topf schmelzen lassen. Je größer desto besser. Die Temperatur ist äußerst wichtig: Sie sollte möglichst niedrig sein. Im Idealfall ist sie genau unter dem Punkt, an dem die Butter und die Zwiebeln anfangen zu zischen. Beim umrühren sollte jodoch Dampf aufsteigen. Die Zwiebeln werden lange nicht den Eindruck machen sich zu verändern. 
		
		Auch wenn kaum Änderungen sichtbar sind, sollten die Zwiebeln umgerührt werden. Abhängig von der genauen Temperatur, der Pfanne, etc. muss das aber nur alle 15-30 Minuten geschehen. %HINWEIS: Deckel bei Pfanne erhöht die Temperatur -> öfter rühren (a pan for example will lose more heat and thus will require more time to get the onions done but on the other hand you will get more time in between stirings).
		
		%\step The onions are done when they are glassy and -depending on the type of onion- brownish-golden.
		Die Zwiebeln sind fertig, wenn die glasig und -abhängig von der Art der Zwiebel- braun-gold sind. %HINWEIS: Rote Zwiebeln beispielsweise verlieren ihre Farbe
		}
	\mainPicture{pictures/onions.jpg}{Rote Zwiebeln während des Schmorens}
	
\end{cleanrecipe}