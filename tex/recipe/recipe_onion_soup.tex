% Complete recipe example
\begin{recipe}
[% 
    preparationtime = {\unit[20]{min}},
    bakingtime={\unit[3.5-4.5]{h}},
    portion = {\portion{1}}
]
{Red Onion Soup}
    
    \graph
    {% pictures
    }
    
    \introduction{%
    	
    }
    
    \ingredients{%
       \unit[500]{g} & Red Onions\\
        & Butter\\
       \unit[500]{ml} & Red or White wine (to taste)\\
       \unit[2(?)]{cups} & dark or light Balsamico Vinegar\\
        & Kraeuter der Provence
    }
    
    \preparation{%
        \step Stir the onions as described in ''recipe link''. If you stirred the onions in a pan, tranfer them now into a pot. Be sure to wash everything out of the pan, as you want every last bit of the onions, most importantly the juice in your soup. Also add two cups of water. If you have done everything right, after a bit of stirring the soup should already be non transparent.
        
        \step Now raise the temperature (the soup should be boiling) and add the wine. Cook the soup until the alcohol is completly gone. For reasons described in the chemestry part of the basics chapter, you should rather cook it a little to much than to little. 
        
        \step Only after all alcohol is vanished, you should add the vinegar and again, to compensate for the loss of water, 2 cups of water.
    }
    
\end{recipe}