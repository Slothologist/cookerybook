\section{Basics}
In this part of the book we will cover some basic knowledge and skills.

First we will discuss more general kitchen knowledge. %luca ergänz ma

In the second part, we will be focussing on food and its properties. That includes e.g. how to spot an ripe avocado, or what ingredients you could substitute with another. It should give you a more detailed view to one of the most crucial parts of cooking: The raw materials you use.

We will then take a look on the more science oriented side of things. Don't let the titel ``Chemistry'' fool you, a science degree won't be needed to understand this chapter. We will however take a look of the do's and dont's in regard to wanted -and unwanted- chemicl reactions that may occur in your kitchen.

In the end, we will then take a look at the first recipes. Those will not  provide you with a yummy and good meal, but they are for key components most other recipes rely on. We will for example see how onions are meant to be stewed and how noodles should be cooked.

\subsection{General Kitchen Knowlegde}

\subsubsection{How to get rid of fruit flies}

Who doesn't know and hate them: Fruit flies. Every year in the summer (and to some extend in the spring and fall to) they tend to swarm our kitchens and feast on our fruits. In case you don't want them to conquer your kitchen, here is a simple and cheap way to get rid of them:

What you will need:
\begin{itemize}
	\item a small bowl, maybe 10cm in diameter
	\item fruit juice, the sweeter the better. If you do not have any, or yours does not seem to wark, sugar is a good substitute
	\item water
	\item dish soap
\end{itemize}
You will want to fill the bowl with water and the juice (or sugar), so that the liquid is not too thick, it should be as close to water as possible. Then drop a bit of the dish soap into the bowl, stir for a bit and place it near your fruits, bio waste or whereever the most fruit flies seem to be. What will happen is that the fruit flies will try to drink the yummy sweet juice you gave them, but due to the absence of water stress the flies will not be able to walk on the water (what they normally could) and drown.

\subsection{Food Knowledge}

\subsection{Chemistry}

\subsection{Basic Ingredients}